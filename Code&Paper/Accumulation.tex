%%%%%%%%%%%%%%%%%%%%%%%%%%%%%%
%%%%%%%%%%%%% Outline %%%%%%%%%%%%%
%%%%%%%%%%%%%%%%%%%%%%%%%%%%%%

% Word target: 7,500 words (currently 7,465)

% To do:
% (1) Results:
%	- Confidence intervals
%	- Gini graphs?

%%%%%%%%%%%%%%%%%%%%%%%%%%%%%%
%%%%%%%%%%%% Preamble %%%%%%%%%%%%%
%%%%%%%%%%%%%%%%%%%%%%%%%%%%%%

% Declare document class and miscellaneous packages
\documentclass[english]{article}
\usepackage{natbib}
\usepackage{amsmath}
\usepackage{mathrsfs}
\usepackage{amssymb}
\usepackage{mathtools}
\usepackage{ctable}
\usepackage{setspace}
\usepackage{longtable}
\usepackage{url}
\usepackage{moredefs,lips} 
\usepackage{IEEEtrantools}
\usepackage{multirow, bigdelim}
\usepackage{enumerate}
\usepackage[normalsize]{caption}
\usepackage{afterpage}
\usepackage[all]{nowidow}
\usepackage{listings}
%\usepackage{fullpage}
\urlstyle{rm}

%Hyper-references
\usepackage{hyperref}
\hypersetup{colorlinks, citecolor=black, filecolor=black, linkcolor=black, 
urlcolor=black, pdftex}

% Title page
\title{Land Accumulation Dynamics in Developing Country Agriculture}
\author{
Heath Henderson, Leonardo Corral, Eric Simning\\
\textit{Inter-American Development Bank} \\
\\
Paul Winters \\
\textit{American University}
\\ \\
}

\date{\today}

\begin{document}

%%%%%%%%%%%%%%%%%%%%%%%%%%%%%%
%%%%%%%%% Title Page and Abstract %%%%%%%%%
%%%%%%%%%%%%%%%%%%%%%%%%%%%%%%

\begin{titlepage}
\maketitle

\begin{abstract}
Inegalitarian distributions of agricultural landholdings in developing countries
have been found to diminish economic growth, mitigate the poverty-reducing
effects of existing growth, and have adverse effects on agricultural 
productivity.
While the consequences of land inequality are relatively well understood, the 
increasing prominence of land rental and sales markets has raised 
important questions regarding causes of distributional outcomes when 
private initiative is predominant.
With unique panel data from Paraguay for the years 1991, 1994, 1999, 2002, 
and 2007, we thus employ a generalized method of moments (GMM) 
estimator for dynamic panel models in an effort to simultaneously test the 
leading theories of such land accumulation dynamics. 
The results suggest that farm growth indeed varies systematically with 
select observable characteristics (i.e.\ land operated and titled area), 
which implies a formal rejection of stochastic growth theories (i.e.\
Gibrat's Law). 
Furthermore, the estimates suggest that a dualistic agrarian structure is 
the likely product of the unfettered operation of land markets, though
land titling interventions may possess the capacity to reduce the rate at 
which inequality manifests. \\
\\
\textit{Key words}: Dynamic panel models; land accumulation; 
land inequality; Paraguay \\
\textit{JEL codes}:  C23; O13; O54; Q15
\end{abstract}
\thispagestyle{empty}
\end{titlepage}
\newpage

% Set citations without comma between author and year
\bibpunct{(}{)}{;}{a}{}{,}

\doublespacing

%%%%%%%%%%%%%%%%%%%%%%%%%%%%%%
%%%%%%%%%%% Introduction %%%%%%%%%%%%
%%%%%%%%%%%%%%%%%%%%%%%%%%%%%%

\section{Introduction}
\label{sec: intro}

The distribution of agricultural landholdings in developing countries is a 
relatively well-documented source of economic inefficiency.
Whether due to multiplier effects \citep{mellor1976}, credit rationing 
\citep{deininger1998}, or fostering extractive institutions 
\citep{acemoglu2002}, inequality in the distribution of land has been linked 
to diminished economic growth.
Such inequality has also been found to mitigate the poverty-reducing effects 
of existing growth, as asset-impoverished households commonly lack the 
capacity to make productive investments 
\citep{deininger1998, ravallion2002, lipton2009}.
Furthermore, due to the often-observed productivity advantage of small-scale 
agricultural producers, it has been demonstrated that inegalitarian distributions 
of land have adverse effects on agricultural productivity  
\citep{eswaran1986, vollrath2007, lipton2009}.

While documentation of the consequences of inequality is prevalent, the causes 
are less well understood.
Historically, the allocation and reallocation of agricultural landholdings in 
developing countries has indeed been driven by inheritance and/or 
administrative processes (e.g.\ land reform) 
\citep{binswanger1995, deininger2001, lipton2009}.
The increasing prominence of land rental and sales markets, however, raises 
questions regarding the forces driving land accumulation (or decumulation) 
when private initiative is predominant
\citep{boucher2005, deininger2008b, holden2009}.%
\footnote{It is important here to clarify our use of the term ``accumulation.''
We define land accumulation as ``the acquisition or gradual gathering of land 
use rights or land access for purposes of agricultural production.''  
Notably, this includes expansion of the farm unit via legal (e.g.\ ownership, 
rental, or sharecropping arrangements) or extralegal (e.g.\ squatting) means.
While we use the terms land accumulation and farm growth synonymously, 
land accumulation appears less ambiguous as farm growth can occur along
multiple dimensions (e.g.\ growth in the quantity or value of output, capital
accumulation, employment increases, etc.).}
Theories of such land accumulation dynamics have invoked stochastic growth 
processes, factor market imperfections, or institutional/legal considerations, 
among others, in attempting to explain observed distributional outcomes.
Yet, with few exceptions, such theories have not been subject to adequate
empirical scrutiny.

For a particularly striking example of this assertion, consider one of the 
most influential theories of firm (or farm) growth dynamics, which is known 
as ``Gibrat's Law of Proportionate Effects.''
Gibrat's Law posits that firm growth is a stochastic process, operating 
independently of firm size, and the limiting distribution of firm size is 
log-normal \citep{gibrat1931, sutton1997}.
The theory has received considerable empirical attention in the context
of developed country agriculture.
For example, \citet{jarrett1968}, \citet{shapiro1987}, \citet{weiss1999}, 
and \citet{melhim2009a} rejected Gibrat's hypothesis for the cases of Australia, 
Canada, Austria, and the United States, respectively, whereas \citet{clark1992}, \citet{fulton1995}, and \citet{melhim2009b} found evidence for the  
theory when using data from Canada, Canada, and the United States, 
respectively.  
In stark contrast, it appears that \citet{shergill1991} -- who found support for 
Gibrat's Law for the case of India -- is the only empirical analysis to date set 
within the context of developing country agriculture.

Regarding institutional/legal considerations, a more nuanced example can be 
found in the empirical literature exploring the economic effects of land 
tenure security.
As well-defined property rights mitigate expropriation risk, facilitate gains 
from trade, and support financial market transactions it is theorized that
tenure security promotes investment in and the efficient use of physical and 
human capital \citep{besley2010}.
A number of studies have indeed found positive effects of land formalization 
on land-related investments \citep{feder1988, besley1995, deininger2008} and 
land market activity \citep{deininger2003, boucher2005, deininger2008b}.%
\footnote{It is important to note that such positive effects are by no means
a universal finding. 
Several studies have found that land-related investment \citep{migot1991, 
gavian1996, brasselle2002} and land market activity \citep{deininger2005, 
gould2006, barnes2007} are not appreciably affected by land formalization.}
While these findings are suggestive of the fact that tenure security exerts 
influence on the distribution of agricultural landholdings, it nevertheless 
appears that the link has yet to be conclusively examined and quantified.
\citet{boucher2005}, for example, explored distributional outcomes before 
and after land market liberalization in Honduras and Nicaragua in the 1990s, 
but the land formalization effect was not uniquely identified as formalization 
initiatives were but one component of the reforms.

In this context, we seek to further the literature examining the causes of 
inegalitarian distributions of agricultural landholdings in developing 
countries.
We specifically examine the case of Paraguay, which represents a particularly 
appropriate setting due to the country's history of land inequality and conflict 
as well as the scope of its liberalization efforts.
With unique panel data for the years 1991, 1994, 1999, 2002, and 2007, we thus 
employ a generalized method of moments (GMM) estimator for dynamic panel 
models in an effort to simultaneously test the leading theories of land 
accumulation dynamics.
The results of the analysis suggest that farm growth indeed varies 
systematically with select observable characteristics 
(i.e.\ land operated and titled area), which implies a formal rejection of 
Gibrat's Law. 
Furthermore, the estimates suggest that a dualistic agrarian structure is the 
likely product of the unfettered operation of land markets, though land titling 
interventions may possess the capacity to reduce the rate at which inequality 
manifests.

The rest of the paper is organized as follows: Section \ref{sec: theories} 
elaborates upon the leading theories of land accumulation dynamics, 
Section \ref{sec: data} provides background on Paraguay as well as discusses 
the available data, Section \ref{sec: methodology} outlines the methodological 
approach and empirical model, Section \ref{sec: results} describes the results 
of the analysis, and Section \ref{sec: conclusions} concludes.

%%%%%%%%%%%%%%%%%%%%%%%%%%%%%%
%%%%%%%% Theoretical Considerations %%%%%%%%%
%%%%%%%%%%%%%%%%%%%%%%%%%%%%%%

\section{Theoretical Considerations}
\label{sec: theories}

In this section we elaborate upon the leading theories of farm growth or 
land accumulation dynamics.  
The theories considered are six-fold and are classified by the primary 
phenomenon invoked: (1) stochastic growth processes; 
(2) factor market imperfections; (3) institutional/legal considerations; 
(4) the life cycle hypothesis; (5) heterogeneous managerial experience; and 
(6) differential human capital. 
In what follows we discuss each of these theories in turn, providing formal
treatment where possible.

Beginning with stochastic growth processes, one of the earliest and most 
influential of such theories, as mentioned, is known as ``Gibrat's Law of 
Proportionate Effects.'' 
Put forth by Robert Gibrat in his work 
\emph{In\'{e}galit\'{e}s \'{E}conomiques} (\citeyear{gibrat1931}), 
the theory attempts to explain the widespread appearance of skew 
distributions, most notably with respect to firm or farm size.
To illustrate Gibrat's Law, let $x_t$ denote firm size at time $t$ and the 
random variable $\varepsilon_t$ denote the proportional rate of growth such 
that $x_t = x_{t-1}(1 + \varepsilon_t) = x_{0}(1 + \varepsilon_1)(1 + 
\varepsilon_2) \ldots (1 + \varepsilon_t)$ or $\log(x_t) \approx \log(x_{0}) 
+ \varepsilon_1 + \varepsilon_2 + \ldots + \varepsilon_t$. 
Under the assumption that $\varepsilon$ is i.i.d.\ with mean $\mu$ and 
variance $\sigma^2$, as $t \to \infty$ the distribution of $\log(x_t)$ is 
approximately normal with mean $t \cdot \mu $ and variance 
$t \cdot \sigma^2$. 
Gibrat thus contended that firm growth $g_t \equiv \log(x_t) - \log(x_{t-1}) 
\approx \varepsilon_t$ is a stochastic process and the limiting distribution of 
firm size is log-normal.
Most importantly, the central testable hypothesis is that firm growth is 
independent of initial firm characteristics, most notably firm size 
\citep{sutton1997}.%
\footnote{In the wake of \emph{In\'{e}galit\'{e}s \'{E}conomiques}, 
a wealth of empirical literature has emerged seeking to test Gibrat's Law, 
much of which has focused on the agricultural sector and farm size growth. 
Such empirical tests typically consist of estimating some variant of the 
following: $\ln(x_{i,t}) - \ln(x_{i,t-1}) = \alpha + \beta \ln(x_{i,t-1}) + 
u_{i,t}$ where $x_{i,t}$ represents the size of farm $i$ at time $t$, $\alpha$ 
and $\beta$ are parameters to be estimated, and $u_{i,t}$ is the error term. 
Rejection of the null hypothesis $\beta=0$ entails a rejection of Gibrat's Law \citep{weiss1999}.} 

Regarding factor market imperfections, \citet{carter1993} developed a theory 
of land market competitiveness whereby a systematic relationship between 
farm growth and farm size manifests.
On the basis of exogenously-given land endowments, utility-maximizing 
agents choose their optimal time allocation 
(i.e.\ on-farm and off-farm labor) and purchased inputs 
(i.e.\ hired labor and fertilizer usage) in the presence of labor and 
capital market imperfections.%
\footnote{More specifically, labor market imperfections entail that hired 
labor requires supervision and agents who seek off-farm employment face a 
distinct probability of unemployment. 
Credit market imperfections entail that the quantity of working capital 
available to a given agent depends on that agent's land endowment.}
Let $\pi(T)$ be the optimal value function where $T$ is the land endowment.%
\footnote{For the sake of brevity, all other arguments in $\pi(\cdot)$ are 
suppressed.}
The reservations price for $\varepsilon$ additional hectares of land is then 
$\rho(T) = \sum_{t=1}^{\infty} \Delta_t(T)/[1 + \mu(T + \varepsilon)]^t$
where $\Delta_t(T) = [\pi(T + \varepsilon) - \pi(T)]/\varepsilon $ and 
$\mu(T + \varepsilon)$ is the shadow price of capital.
The authors found that the smallest farm units witness relatively high 
reservation prices due to their high marginal unemployment in the labor 
market. 
Medium-sized farms also demonstrated high reservation prices due to 
their ability to mediate labor and capital market imperfections.
Therefore, as reservation prices are expected to be highly correlated with farm 
growth rates, it is contended that farm size is an important determinant of 
farm growth, though the relationship may be highly non-linear.%
\footnote{In other words, ``the model identifies an agrarian structure 
composed of mid-sized farms, and poverty refuge \emph{minifundias} as a 
likely outcome of the unfettered operation of the land market'' 
\citep[pg.\ 1097]{carter1993}.}

With respect to institutional/legal considerations, \citet{carter1998b} 
incorporated notions of tenure insecurity into a land market competitiveness 
model similar to that described above.
Letting $0 < \phi < 1$ denote the single-period probability that a given 
household is dispossessed of its land, the reservation price of land becomes 
$\rho(T) = \sum_{t=1}^{\infty} [(1 - \phi)^t \Delta_t(T, \phi)]/
[1 + \mu(T + \varepsilon, \phi)]^t$.
Tenure insecurity affects the reservation price formulation through three distinct
channels: (1) the term $(1 - \phi)^t$ introduces uncertainty-based discounting 
of future earnings; (2) the presence of $\phi$ in $\Delta_t(\cdot)$ suggests 
that tenure insecurity may depress incremental earnings from land by affecting 
factor allocations; and (3) the incorporation of $\phi$ in $\mu(\cdot)$ reflects 
the fact that tenure insecurity may have credit supply effects due to the 
collateralizability of land.
Thus, it is evident that, all else equal, tenure insecurity reduces incentives to 
land accumulation as $\partial \rho / \partial \phi < 0$.
There may, however, be important interaction effects between tenure insecurity
and land endowments as credit markets may persistently exclude the 
lesser-endowed regardless of the legal collateralizability of their land 
\citep{carter1988}.

Turning to the life cycle hypothesis, \citet{chayanov1966} was among the first 
to suggest that land accumulation is intimately tied to the growth of the 
individual family. 
Stated simply, assuming the absence of a well-functioning labor market, 
Chayanov suggested that farm size passively adapts to the equilibrium level of 
income of a given agricultural household, which is determined by balancing 
the marginal utility of income and the marginal drudgery of labor. 
The location and shape of these curves was said to be heavily influenced by 
family size and composition, as the marginal utility of income depends upon 
family consumption demands and the marginal drudgery of labor hinges upon 
the size of the family work force. 
In traversing the family life cycle,%
\footnote{The life cycle was said to begin with the marriage of the nuclear 
couple, then proceed through child-bearing and rearing, the entrance of the 
children into the family work force, and finally end with the exit of the children 
from the household to form families of their own.} 
the family initially witnesses increasing consumption demands due to the 
augmentation of family size, which induces a steady upward shift in the 
marginal utility of income curve. 
However, as the children become of working age, there then appears a 
downward shifting of the marginal drudgery of labor curve due to the reduced 
degree of labor intensity per worker. 
The interaction of these forces, then, generates a persistently increasing 
equilibrium level of income and, thus, farm size 
\citep{harrison1975, banaji1976}. 

Discussion of the relationship between managerial experience and firm 
growth generally centers on the learning model put forth in 
\citet{jovanovic1982}. 
In the model, at time $t$, firms choose their output level $q_t$ so as to 
maximize expected profits $p_t q_t - c(q_t)x_t^*$ where $p_t$ is the 
exogenously-given output price, $c(q_t)$ is the cost function, and $x_t^*$ 
denotes the expectation of $x_t$, which is a random variable capturing 
efficiency considerations. 
For a firm of type $\theta$, $x_t = \xi (\eta_t)$ where $\xi (\cdot)$ is a 
positive, strictly increasing, and continuous function, and 
$\eta_t = \theta + \varepsilon_t$ where $\varepsilon_t \sim N(0,\sigma^2)$. 
While $\theta$ is unknown to a given firm, the distribution of $\theta$ across 
firms is known. 
Further, $\eta_t$ can be inferred by observing costs at time $t$. 
Letting $n$ be the age of a given firm and 
$\bar{\eta}_n = \sum_{i=1}^n \eta_i/n$, we can then write 
$x_t^* = \int \xi(\eta)P^0(\hspace{1pt} \cdot \hspace{1pt} | 
\hspace{1pt} \bar{\eta}_n, n)$ where $P^0(\hspace{1pt} \cdot 
\hspace{1pt} | \hspace{1pt} \bar{\eta}_n, n)$ is the normal posterior 
distribution of $\eta_t$, the variance of which only depends on $n$. 
It is thus clear that $x_t^*$ converges to a constant as firms age, which implies 
an equilibrium scale of production for mature firms. 
While younger firms have more variability in growth rates, it can also be 
shown that they will grow faster, as Jovanovic demonstrated that growth is an 
increasing function of $x_t^*/x_{t+1}^*$ and $E(x_t^*/x_{t+1}^*)>1$. 
It is hypothesized then that there exists an inverse relationship between 
firm growth and firm age.

Finally, the effects of human capital on land accumulation can be understood
in terms of the structural evolution model put forth in \citet{rodgers1994}.%
\footnote{See \citet{sumner1987} for an alternative, albeit similar, theoretical
model.}
Agents in the model have two human capital attributes, $x$ and $y$, where
$x$ represents agriculture-specific human capital and $y$ represents general 
human capital.%
\footnote{Agriculture-specific human capital $x$ is assumed to be primarily 
determined by learning-by-doing, though formal education may also play an 
important role. 
General human capital $y$ is assumed to be determined by formal education, 
employment history, and inherent ability.}
On the basis of such endowments, agents then choose whether to engage in 
agricultural production or off-farm employment.
Off-farm income $w$ is assumed to be an increasing function of general
human capital (i.e.\ $\partial w/ \partial y > 0$). 
Agricultural income $m$ is determined by choosing land $d$ and purchased 
inputs $k$ to maximize profits $p x F(d,k) - rd - vk$ where $p$ is the price 
of agricultural output, $F(\cdot)$ is a standard production function, 
and the unit prices of $d$ and $k$ are $r$ and $v$, respectively.
Given that the marginal products of $d$ and $k$ are increasing functions of 
$x$, agents with relatively high $x$ will choose to farm as $m > w$.
Agents with relatively high $y$, however, will choose off-farm employment
 as $w > m$.
The equilibrium distribution of land is then driven by the distribution of the 
two types of human capital across agents.
More interestingly, agricultural producers with relatively high $x$ likely 
grow faster as technology adoption costs may vary inversely with human 
capital levels.%
\footnote{Rodgers also suggested that agricultural producers with relatively 
high $x$ may also grow faster due to the fact that they are able to spread fixed 
technology adoption costs over a greater quantity of output.}

%%%%%%%%%%%%%%%%%%%%%%%%%%%%%%
%%%%%%%%%%%%%% Data %%%%%%%%%%%%%
%%%%%%%%%%%%%%%%%%%%%%%%%%%%%%

\section{Paraguay Background and Data}
\label{sec: data}

Before discussing the methodological approach used to examine the alternative
hypotheses considered in the previous section, it is beneficial to elaborate upon 
the relevance of the Paraguayan setting as well as the available data.
As such, with special emphasis on the distribution of agricultural landholdings, 
this section first provides an overview of Paraguay's agricultural sector. 
After this contextual discussion, we then discuss the data collection process 
and descriptive statistics associated with the unique panel data set used in the 
empirical analysis.

Paraguay, with a gross domestic product per capita of \$2,967 and a poverty 
rate of 38 percent, is among the poorest countries in Latin America. 
Moreover, economic growth in Paraguay is intimately tied to the agricultural 
sector, as agriculture accounts for 24 percent of gross domestic product and 
27 percent of the country's employment. 
Food exports, in addition, represent 87 percent of total merchandise exports 
\citep{wdi2012}.%
\footnote{The data presented above pertains to the year 2008.} 
With an estimated Gini coefficient of 0.93, the distribution of landholdings in 
Paraguay is one of the most inegalitarian in the world \citep{lipton2009}. 
On the one hand, just over 63 percent of producers operate landholdings less 
than ten hectares, but account for just over two percent of total farming area.  
On the other hand, under two percent of producers operate landholdings 
greater than 1,000 hectares and account for nearly 80 percent of total farm 
land. 
Compounding issues of land inequality is the existence of pervasive tenure 
insecurity, as approximately 27 percent of producers do not have lawful 
permission to use the land which they operate. 
Moreover, of those producers operating less than five hectares, 36 percent are 
classified as illegal squatters \citep{mag2012}. 

Unlike much of Latin America, the development of Paraguay's 
\emph{latifundia-minifundia} system is not primarily rooted in $16^{th}$ 
century European colonization.
The concentration of landholdings and marginalization of 
\emph{campesinos} is instead largely grounded in the aftermath of the War 
of the Triple Alliance (1864-1870), whereby large tracts of state land were sold 
to foreign investors and Paraguayan elite as a means to settle war debt.
In the $20^{th}$ century, this dualistic agrarian structure was then exacerbated 
by population growth, a lack of opportunities outside of
agriculture, and a system of partial inheritance that induced further 
\emph{minifundia} fragmentation, most notably in Paraguay's central region
\citep{baer1984, danielsen2009}.
The Stroessner regime (1954-1989), viewing the resulting inegalitarian 
distribution of land as a source of productive inefficiency and rural unrest, 
thus enacted the Agrarian Statute of 1963 and embarked on a large-scale 
colonization program with the stated intent to increase rural welfare.
In practice, however, land was primarily distributed to associates of the 
regime (e.g.\ armed forces, rural elite, government officials, etc.), which led to 
the replication of the \emph{latifundia-minifundia} system in the 
colonization areas \citep{weisskoff1992, nagel1999}.

The fall of the Stroessner regime in 1989 created space for the expression of 
rural discontent, which was manifest primarily in a wave of land occupations.
As such, in the ensuing democratization and liberalization process, issues of  
land inequality and tenure insecurity became firmly established on the national 
political agenda.
As a reflection of the continued strength of the rural elite, however, the new 
constitution of 1992 excluded the traditional usufruct right to land, provided a 
strong guarantee to property rights, and generally left land redistribution to 
market forces \citep{nagel1999, danielsen2009}.
With continued \emph{campesino} unrest and renewed calls for agrarian 
legislation, the Agrarian Statute of 2002 was then enacted after a protracted 
period of political deadlock.
The new legislation was nonetheless technical rather than redistributive in 
character, and the stated aim of rural development and poverty alleviation 
was to be achieved through increased productivity, the stimulation of
agro-industry, and overall reduction of market interventions.
Thus, legislation in the post-Stroessner era reflects the emergence of liberal 
democratic values and a shift toward neoliberal economic policies
\citep{danielsen2009}. 

For the years 1956 to 2008, Tables \ref{distf} and \ref{distl} provide 
information on the distribution of farms and landholdings by farm size 
categories, respectively. 
Looking at the data for 1956, it is evident that the Stroessner regime 
inherited considerable inequality as nearly 70 percent of producers 
operated farms less than ten hectares, but accounted for just over two 
percent of the total land cultivated.
Conversely, in that same year, just over three percent of producers operated
farms greater than 100 hectares, but accounted for nearly 93 percent
of the total land cultivated.
Largely due to the regime's land colonization programs, it is apparent that 
the number of farms and area cultivated expanded greatly by 1991. 
The inegalitarian distribution of landholdings, however, remained intact, as in 
that year the largest four percent of producers operated 88 percent of total 
land cultivated.
Finally, the liberalization process that began after 1991 appears to have been
accompanied by a further concentration of landholdings, as the percentage of 
land operated by producers with less than ten hectares decreased from 
approximately three to two percent whereas that operated by producers with 
more than 100 hectares increased from 88 to over 92 percent.

As gaining insight into the micro-level forces inducing distributional changes 
throughout the liberalization period is greatly facilitated by panel data, it is thus 
beneficial to turn to the discussion of the data used in the empirical analysis.
In 1991, the Land Tenure Center (LTC) at the University of 
Wisconsin-Madison and the \emph{Centro Paraguayo de Estudios 
Sociol\'{o}gicos} (CPES) administered surveys to 300 rural Paraguayan 
households, which were selected in accordance with a stratified, multi-stage 
random sampling framework. 
The sample was distributed across three regions of Paraguay: (1) the traditional 
``\emph{minifundia}'' zone located in the department of Paraguar\'{i}, 
which is characterized by small plots and low soil fertility, but possesses a 
favorable proximity to the country's largest cities; (2) the colonization zone in 
the department of San Pedro, which is characterized by higher quality soils, 
fewer land conflicts, but lacking infrastructure; and (3) the frontier region 
located in the department of Itap\'{u}a, which is characterized by the best 
land, the highest rainfall, and larger farms employing modern technology.%
\footnote{These regions were selected primarily because they are precisely 
those regions where much of the country's agricultural production and 
land scarcity problems are concentrated.}
Within these regions, the sample was further stratified by household land 
endowments (0-5, 5-10, 10-20, 20-50, and $>$50 hectares).

The LTC-CPES survey is panel in nature and was again administered in the 
years 1994, 1999, 2002, and 2007. 
For a variety of reasons, issues of attrition included, households were 
strategically added to the sample in select years.
As such, the panel is unbalanced with 300, 284, 293, 223, and 446 reliable 
observations in the years 1991, 1994, 1999, 2002, and 2007, respectively. 
Importantly, 139 households were successfully surveyed in each of the five 
years, 70 were surveyed in four of the years, 41 in three of the years, 65 in two 
of the years, and the remaining households were only surveyed once. 
While it is evident that the number of households interviewed in each survey 
year is relatively small, such shortcomings are partially compensated by the 
depth of the interviews administered. 
Particularly relevant at present are the detailed modules on modes of land 
access, property rights, household characteristics and individual-level 
demographic traits, as well as production and income. 
For further information regarding the LTC-CPES survey see 
\citet{fletschner2002}, \citet{carter2003}, or \citet{schechter2007}.

Table \ref{def} provides definitions for all variables utilized in the analysis 
and Table \ref{desc} presents descriptive statistics. 
The variables defined follow directly from the discussion in Section 
\ref{sec: theories}.%
\footnote{See Section \ref{sec: methodology} for a detailed 
discussion of the relationship of each variable to the hypotheses
put forth in Section \ref{sec: theories}.}
As the definitions and descriptive statistics are largely self-explanatory,%
\footnote{It is worth nothing that in Table \ref{desc} the means of land 
operated and titled area are of a noticeably lower magnitude for 2007. 
At first glance this might appear to be due to the addition of new units to the 
sample, but when looking at the same variable means after omitting new 
observations it is apparent that the differential persists, albeit to a lesser 
extent.}
we focus our attention on Figure \ref{fig: land}, which offers an alternative 
perspective on changes in farm sizes across the survey period.
Using the balanced sub-panel, we divided the sample in the baseline period 
into four mutually exclusive groups according to land operated (see legend for 
categorization scheme), and then plotted mean land operated for each cohort 
in each survey year.
Three patterns are worth noting.
First, the degree of variability in each series is particularly interesting, as it is 
suggestive of non-negligible land market activity and potentially interesting
accumulation dynamics.
Second, while the 5-10 hectare cohort seemingly stagnated throughout the 
survey period, the other groups witnessed episodes of considerable growth.
The 10-20 hectare cohort, for example, witnessed an approximate doubling of 
mean farm size from 1999 to 2007.
Finally, mean farm size of the $<$5 hectare cohort actually surpassed that of the 
5-10 hectare cohort by 2002 and remained (slightly) greater in 2007.

%%%%%%%%%%%%%%%%%%%%%%%%%%%%%%
%%%%%%%%%%% Methodology %%%%%%%%%%%%
%%%%%%%%%%%%%%%%%%%%%%%%%%%%%%

\section{Methodology}
\label{sec: methodology}

The previous section demonstrated that legislation in Paraguay's 
post-Stroessner era created space for private initiative to stimulate change in 
the distribution of agricultural landholdings.
It was further demonstrated that the era was characterized by non-negligible 
land market activity and potentially interesting accumulation dynamics.
In light of these observations, this section considers the methodological 
approach and empirical model associated with our evaluation of the 
theories discussed in Section \ref{sec: theories}.

While our specific empirical model is discussed in detail below, it is first 
instructive to consider, in a generalized context, estimation of the following 
autoregressive-distributed lag model: 
\begin{equation}
y_{i,t} = \beta_1 y_{i,t-1} + \beta_2 x_{i,t} + \alpha_i +  u_{i,t}
\label{adl}
\end{equation}

\noindent
where, for producer $i=1,2,\ldots,N$ at time $t=1,2,\ldots,T$, $y_{i,t}$ 
denotes some firm size measure, $x_{i,t}$ is a vector of current or lagged 
values of additional explanatory variables, $\alpha_i$ is the producer-specific 
effect, and $u_{i,t}$ is the error term, which is assumed to be serially 
uncorrelated and independent across producers. 
Finally, $\beta_1$ and $\beta_2$ represent parameters to be estimated
\citep{bond2002}.

Estimation of Eq.\ (\ref{adl}) via the within-groups estimator is 
inconsistent as the requisite transformation introduces a correlation between 
the transformed $y_{i,t-1}$ and the transformed error term. 
First-difference and orthogonal deviations transformations can also eliminate 
the individual effects $\alpha_i$, but in both cases the correlation between the 
transformed $y_{i,t-1}$ and transformed error term persists.%
\footnote{In contrast to the first-difference transformation in which a lagged 
observation is subtracted from the contemporaneous observation, the 
orthogonal deviations transformation subtracts the average of all future 
available observations. 
The primary benefit of using orthogonal deviations over differencing is that 
data loss is minimized in unbalanced panels \citep{arellano1995}.} 
However, as opposed to the within-groups estimator, the first-difference and 
orthogonal deviations transformations do not introduce all realizations of the 
disturbances into the transformed equation, thereby implying that further lags 
of the explanatory variables are available to be used as instruments. 
For example, in the orthogonal deviations case, recalling that the disturbances 
are assumed to be serially uncorrelated and further assuming that initial 
conditions are predetermined,%
\footnote{That is, $y_{i,1}$ is uncorrelated with subsequent disturbances 
$u_{i,t}$ for $t=2,3,\ldots,T$.} 
$y_{i,t-2}$ is uncorrelated with $u_{i,t}^{\perp} = c_t\left[u_{i,t} - 
\frac{1}{T - t }\left(u_{i,t + 1} + \ldots + u_{i,T} \right) \right]$,%
\footnote{In the orthogonal deviations expression, $c_t = 
\sqrt{(T - t)/(T - t + 1)}$ is introduced to equalize the variances.} 
and the 2SLS estimator is consistent in large $N$, fixed $T$ panels 
\citep{anderson1982, arellano1995, bond2002, roodman2009}.

While the 2SLS estimator is consistent it is not asymptotically efficient as it 
does not utilize all available moment conditions or account for the 
transformed nature of the error term. 
\citet{arellano1991} developed a generalized method of moments (GMM) 
estimator for dynamic panel models in an effort to remedy the shortcomings 
of the 2SLS approach. 
The authors noted that, for $T>3$, additional instruments are available as, 
for example, $y_{i,t-2}$ and $y_{i,t-3}$ can be used as instruments for the 
transformed equation when $t=4$.
In the context of a simple autoregressive model (i.e.\ $\beta_2=0$), the 
instrument matrix can be written as follows:
\begin{equation}
Z_i = \left[ 
\begin{array}{ccccccc}
y_{i,1} & 0 & 0 & 0 & 0 & 0 & \cdots   \\\
0 & y_{i,1} & y_{i,2} & 0 & 0 & 0 & \cdots   \\\
0 & 0 & 0 & y_{i,1} & y_{i,2} & y_{i,3} & \cdots  \\\
\vdots & \vdots & \vdots & \vdots & \vdots & \vdots & \ddots
\end{array}
\right]
\label{z}
\end{equation}

\noindent
or ``collapsed'' as 
\begin{equation}
\left[ 
\begin{array}{ccccccc}
y_{i,1} & 0 & 0 & \cdots   \\\
y_{i,1} & y_{i,2} & 0 & \cdots   \\\
y_{i,1} & y_{i,2} & y_{i,3} & \cdots  \\\
\vdots & \vdots & \vdots & \ddots
\end{array}
\right]
\label{zc}
\end{equation}

\noindent
where, for the $i^{th}$ entity, the rows correspond to the transformed 
equation for periods $t=3,4,\ldots,T$.%
\footnote{The primary rationale for collapsing the instrument matrix is to
reduce the instrument count.  
The class of models considered here tend to generate instruments 
prolifically, which can overfit endogenous variables and weaken select
specification tests.
See \citet{roodman2009b} for more information.}
The above is readily generalized to the case of the autoregressive-distributed 
lag model under consideration. 
While the availability of further instruments will depend on the assumptions 
made regarding the correlation between the additional explanatory variables 
and the error term, for illustrative purposes let $x_{i,t}$ be scalar and 
endogenous. 
Accordingly, the vector $(y_{i,1}, \ldots, y_{i,t-2})$ can be replaced by 
$(y_{i,1}, \ldots, y_{i,t-2}, x_{i,1},\ldots,  x_{i,t-2})$ in forming 
each row of the instrument matrix. 

Using orthogonal deviations as the choice transformation, the asymptotically 
efficient GMM estimator exploits the moment conditions 
$E[Z_i' u_i^\perp]=0$ for $i=1,2,\ldots,N$ where $u_i^\perp=
(u_{i,3}^\perp, u_{i,4}^\perp, \ldots , u_{i,T}^\perp)'$ to minimize the 
following criterion:
\begin{equation}
J_N = \left(\frac{1}{N} \sum_{i=1}^N u_{i}^{\perp \prime} Z_i \right) 
W_N \left(\frac{1}{N} \sum _{i=1}^N Z_i' u_i^\perp \right)
\end{equation}

\noindent
where
\begin{equation}
W_N = \left[\frac{1}{N}\sum_{i=1}^N \left(Z_i' \widehat{u}_i^\perp 
\widehat{u}_i^{\perp \prime} Z_i \right)  \right]^{-1}
\end{equation}

\noindent
is the optimal weight matrix and $\widehat{u}_i^\perp$ denotes estimates of 
the corresponding residuals, which are calculated from a preliminary consistent 
estimator. 
As such, the method is known as a two-step GMM estimator. 
The dependence of the two-step weight matrix on estimated parameters, 
however, makes the asymptotic distribution approximations unreliable, and 
thus standard errors are generally calculated using the finite-sample correction 
put forth in \citet{windmeijer2005} 
\citep{arellano1991, bond2002, roodman2009}.

\citet{blundell1998}, building on \citet{arellano1995}, noted that when the 
series in question has near unit root properties, the utilized instruments are 
likely to be weak.
In such situations it may be the case that past changes are more predictive of 
current levels than past levels are of current changes or deviations 
\citep{roodman2009}. 
To fix ideas, reconsider the simple autoregressive case (i.e.\ $\beta_2$=0). 
\citeauthor{blundell1998} suggested utilizing the additional $T-2$ linear 
moment conditions $E[\Delta y_{i,t-1} (\alpha_i + u_{i,t})]$ for 
$i=1,2,\ldots,N$ and $t=3,4,\ldots,T$ where $\Delta y_{i,t-1} = y_{i,t-1} - 
y_{i,t-2}$.%
\footnote{It is assumed here that changes in the instrumenting variables are 
uncorrelated with the fixed effects.}
In exploiting the new moment conditions, the authors developed a ``system'' 
GMM estimator whereby the $T-2$ equations in orthogonal deviations 
(or first-differences) and the $T-2$ equations in levels are stacked and the 
instrument matrix is augmented as follows:
\begin{equation}
Z_{i}^{+} = \left[ 
\begin{array}{ccccccc}
Z_i  & 0 & 0 & 0 & \ldots \\
0 & \Delta y_{i,2} & 0 & 0 & \ldots \\
0 & 0 & \Delta y_{i,3} & 0 & \ldots  \\
0 & 0 & 0 & \Delta y_{i,4} & \ldots  \\
\vdots & \vdots & \vdots & \vdots & \ddots
\end{array}
\right]
\label{znew}
\end{equation}

\noindent
or ``collapsed'' as
\begin{equation}
\left[ 
\begin{array}{ccccccc}
Z_i  & 0  \\
0 & \Delta y_{i,2}  \\
0 & \Delta y_{i,3}  \\
0 & \Delta y_{i,4}  \\
\vdots & \vdots 
\end{array}
\right].
\label{zcnew}
\end{equation}

\noindent
Estimation then consists of minimizing $J_N$ after properly introducing the 
additional moment conditions. 
Further, analogous to the case of the \citeauthor{arellano1991} GMM 
estimator, $Z_{i}^{+}$ readily incorporates additional instruments 
associated with the autoregressive-distributed lag model.

Turning to the empirical model, in line with the above discussion as well as 
the firm growth literature
\citep{evans1987, sleuwaegen2002, rizov2003, dries2004b}, the following 
basic specification is put forth:
\begin{equation}
\frac{\ln (y_{i,t}) - \ln(y_{i,t-1})}{\delta_t} \times 100 = \ln 
\big[g(y_{i,t-1}, x_{i,t-1})\big] + \alpha_i + u_{i,t}
\label{emodel}
\end{equation}

\noindent
where, for producer $i=1,2,\ldots,N$ at time $t=1,2,\ldots,T$, $y_{i,t}$
represents land operated and $\delta_t$ is the number of years between 
survey periods $t$ and $t-1$. 
The left-hand side is thus an approximation of the annual percentage growth 
rate in land operated. 
Regarding the right-hand side, $x_{i,t-1} = (x_{1,i,t-1}, \ldots, x_{5,i,t-1} )$ 
is a vector of lagged values of explanatory variables including titled area, 
labor, dependents, experience, and education, respectively,%
\footnote{See Table \ref{def} for variable definitions.} 
$\alpha_i$ is the producer-specific effect, and $u_{i,t}$ is the error term. 
Approximating the growth function $g(\cdot)$ by a second-order expansion 
in the logs and augmenting the basic model with time dummies results in the 
empirical model to be estimated.%
\footnote{\label{note: zvdum}In situations where scaling is necessary to 
permit logarithmic transformation of zero-valued explanatory variables, 
we employ the dummy variable procedure outlined in \citet{battese1997}, 
as alternative approaches (e.g.\ adding an arbitrarily small constant) can bias 
parameter estimates.
The method is simple: recode all zero values of explanatory variables to one 
and include in the regression a corresponding dummy variable that takes on 
the value of one if that observation was recoded and zero otherwise. 
Note that dummy variables for land operated and experience are not 
necessary as these variables never take on a value of zero.}

Eq.\ (\ref{emodel}) nests tests of all hypotheses discussed in Section 
\ref{sec: theories}. 
First, joint independence of initial producer characteristics from 
subsequent land accumulation would provide support for Gibrat's Law.
Second, appropriate signs and joint statistical significance of the coefficients 
capturing the (non-linear) effect of land operated on land accumulation 
would substantiate the hypothesis put forth by \citet{carter1993}.
Third, a positive and statistically significant coefficient on titled area would 
corroborate the contention of \citet{carter1998b} that tenure security 
incentivizes land accumulation.
Fourth, a positive and statistically significant coefficient on labor 
(dependents) would lend support to the life cycle hypothesis that land 
accumulation is induced via shifting downward (upward) the marginal 
drudgery of labor (marginal utility of income) curve.
Fifth, a negative and statistically significant coefficient on experience 
would substantiate the hypothesis of \citet{jovanovic1982} that younger 
farms grow faster. 
Finally, as \citet{rodgers1994} suggested that both agricultural-specific and
general human capital are influenced by formal education, the effect of 
education on land accumulation is \emph{a priori} ambiguous.%
\footnote{Data limitations preclude distinguishing between 
agricultural-specific and general human capital.}

It is evident that Eq.\ (\ref{emodel}) is simply a special case of the 
autoregressive-distributed lag model previously discussed. 
As such, utilizing orthogonal deviations to minimize data loss due to the 
unbalanced nature of the panel, we employ the 
\citet{arellano1995}/\citet{blundell1998} ``system'' GMM estimator 
outlined above. 
With the exception of the time dummies which are deemed exogenous, all 
explanatory variables are treated as predetermined. 
The appropriateness of the specification rests primarily on the validity of 
four assumptions: (1) the disturbances lack serial correlation; (2) the 
instruments are exogenous; (3) the instrument count is sufficiently small; 
and (4) attrition bias is absent or minimal. 
The first three of these assumptions are examined post-estimation.
We explore the serial independence assumption with the \citet{arellano1991} 
test for first-order serial correlation in the disturbances and the instrumental 
exogeneity assumption with \citet{hansen1982} tests for overidentifying 
restrictions. 
Further, as too many instruments can bias parameter estimates and weaken 
the Hansen test \citep{roodman2009b}, we monitor the instrument count 
throughout and examine the sensitivity of the results to reductions.

Regarding issues of attrition, ``growth rates can only be measured for 
surviving farms (i.e.\ those still operating in period $t$), and since slow 
growing farms are most like to exit the industry, it is easy to see that small, 
fast growing farms can easily be overrepresented in the sample, thus 
introducing potential bias in the results'' \citep[pg.\ 790]{bakucs2009}. 
While the vast majority of the empirical literature to date suggests that the 
resulting bias is negligible \citep{evans1987, hall1987, weiss1999, dries2004},
to gain further insight into this problem we conducted two types of tests for
attrition bias: (1) the \citet{becketti1988} attrition test and (2) the 
\citet{fitzgerald1998} attrition probit.
As it is the primary variable of interest, each test was conducted with three 
alternative specifications of land operated (i.e.\ in levels, logarithms, and 
growth rates), yielding a total of six tests.
In only one such test do we reject the null hypothesis of no attrition bias at 
the five percent level, thereby suggesting that the issue of attrition is likely 
minimal.% 
\footnote{The results of the tests are available upon request.}
Accordingly, we avoid undue complication of the estimation procedure 
and focus on surviving farms.

%%%%%%%%%%%%%%%%%%%%%%%%%%%%%%
%%%%%%%%%%%% Results %%%%%%%%%%%%%%
%%%%%%%%%%%%%%%%%%%%%%%%%%%%%%

\section{Results}
\label{sec: results}

Tables \ref{sysgmm} and \ref{tests} present the results from the 
\citet{arellano1995}/\citet{blundell1998} ``system'' GMM estimator. 
Given that instrument proliferation is a primary concern, we estimate 
three alternative models in order to examine the sensitivity of the results 
to reductions in the instrument count: (1) FM, which is the full model using 
the instrument set defined in Eq.\ (\ref{znew}); (2) CM, which is the 
collapsed model using the instrument set defined in Eq.\ (\ref{zcnew}); 
and CR, which is the collapsed model where the instrument set is further 
restricted to only two-period lags.
Each model is estimated with a full set of time and ``zero value'' dummies 
(see footnote \ref{note: zvdum}).
Note also that the number of instruments utilized, the associated $R^2$ values, 
and the number of observations are provided for each regression at the 
bottom of the corresponding column in Table \ref{sysgmm}. 

 Looking at Table \ref{sysgmm}, as testament to the consistency of the
 of the ``system'' GMM estimator, we can exploit the fact that OLS and 
 within-groups estimates of the coefficient on the (natural log of the) lagged 
 land operated variable ($y$) are biased in opposite directions. 
 That is, the OLS estimator is biased upward and the within-groups estimator
 is biased downward \citep{bond2002}.
 OLS estimation of the model yields a coefficient on the lagged land operated 
 variable of $-51.42$ and within-groups yields a coefficient of $-64.70$.%
 \footnote{Full OLS and within-groups results are available upon request.}
 It is evident that in each of our three models the coefficient on the lagged
 land operated variable falls within this range, which suggests that the 
 ``system'' GMM estimates indeed represent an overall improvement.
 Given reasonable estimates, then, we can proceed to an in-depth 
 examination of the validity of the underlying assumptions as well as probe
 the implications of our estimates for the hypotheses in question.
 To this end, it is beneficial to focus on Table \ref{tests} in which we 
 present a series of related hypothesis tests.
 
Hypothesis tests (1)-(3) in Table \ref{tests} present the results from the 
specification tests discussed in the previous section. 
Hypothesis test (1) displays $z$-scores from the \citet{arellano1991} test 
for first-order serial correlation in the disturbances.
Hypothesis test (2) displays test statistics from the \citet{hansen1982} 
test for overidentifying restrictions.
Hypothesis test (3) presents a difference-in-Hansen test statistic, which 
tests the validity of the instrument subset associated with the levels model
(i.e.\ the differenced instruments introduced in Eq.\ [\ref{znew}]).
Looking at the FM column, it is evident that we fail to reject the null 
hypothesis of no serial correlation for hypothesis test (1), and instrumental
exogeneity for hypothesis tests (2) and (3).
At first glance the specification thus appears appropriate.
However, as mentioned, instrument proliferation can greatly weaken the 
Hansen tests, and both Hansen tests here are associated with an extreme 
$p$-value of $1.00$, which suggests that the instrument count is excessively 
high. 
Proceeding with caution, then, we turn to hypothesis tests (4)-(9) where we 
present estimates of the marginal effect of (the natural log of) each regressor, 
which are evaluated at the sample means.

It is evident that the marginal effects on dependents ($x_3$), experience 
($x_4$), and education ($x_5$) witness no statistical significance, which 
suggests that land accumulation seemingly operates independently of these 
covariates.
Of those marginal effects that witness statistical significance (i.e.\ $y$, $x_1$, 
and $x_2$ or land operated, titled area, and labor, respectively), it appears that 
all signs are in broad accordance with the theoretical notions discussed in 
Section \ref{sec: theories}.
With respect to land operated, the associated marginal effect indicates that, at
the mean, a one percent increase in the quantity of land operated is expected to 
induce a statistically significant $0.15$ (i.e.\ 14.86/100) unit decrease in the 
annual percentage growth rate of operational landholdings. 
This finding implies a formal rejection of Gibrat's Law and is consistent with 
the contention of \citet{carter1993} that smaller farms witness relatively 
high land reservation prices.
Further, it is evident that the positive marginal effects associated with titled 
area and labor, which can be interpreted analogously, corroborate the 
associated theories: an increase in land titled and household labor tend to 
induce land accumulation.%
\footnote{The non-linear effects associated with these theories is further 
explored below. 
It is, however, worth mentioning that we attempted to include higher-order
polynomials in the model, but their insignificance gave way to the more 
parsimonious specification presented.}

Due to the relatively high instrument count in the FM model (275), these
estimates are potentially suspect.
As such, the CM regression utilizes the collapsed instrument set, which
serves to reduce the instrument count to 128.
Looking to the CM column of Table \ref{tests}, we see that we again fail
to reject the null hypotheses for tests (1)-(3).
With $p$-values of 0.63 and 0.28 for Hansen tests (2) and (3), respectively, 
it is possible to proceed with less concern regarding issues of instrument 
proliferation.
Regarding hypothesis tests (4)-(9) in the CM column, then, it is evident that
the marginal effects associated with $x_3$, $x_4$, and $x_5$ remain 
insignificant.
Further, while the marginal effect on titled area ($x_1$) remains positive
and statistically significant, the marginal effects associated with land operated
($y$) and labor ($x_2$) are now statistically insignificant.
Further yet, we see that the magnitude of the marginal effect on titled area
is reduced to 10.62 from 14.83 in the FM regression.
It is thus apparent that the results are moderately sensitive to the instrument
count.
As such, it is beneficial to consider reducing the number of instruments further.

The CR regression, as mentioned, utilizes the collapsed instrument set and 
additionally restricts lags to two periods.
This serves to further reduce the instrument count to 97.
Once again, in no case do we reject the null hypothesis for tests (1)-(3) in the 
CR column in Table \ref{tests}. 
Furthermore, with $p$-values of 0.64 and 0.73 for Hansen tests (2) and (3), 
respectively, it appears possible to take yet greater comfort in the instrument 
set employed.
With respect to the hypothesis tests on the marginal effects, then, we see that 
the effects associated with $x_2$ through $x_5$ remain statistically 
insignificant.  
Moreover, the marginal effect associated with titled area ($x_1$) remains 
statistically significant and of a similar magnitude to the CM regression.
The primary difference between the CM and CR regressions is that the 
marginal effect on land operated ($y$) is now statistically significant, 
though the magnitude of the effect is similar.
Given that the CR regression is our preferred specification and that both 
land operated and titled area witness statistical significance, it is beneficial
to further examine the economic implications of the associated 
coefficients.

Figure \ref{fig: pgrowth} depicts the relationship between the annual 
percentage growth rate in operational landholdings and initial land 
operated.
The figure was constructed by retrieving the predicted values from the 
CR regression model and then regressing those predicted values on 
initial land operated using a fractional polynomial regression 
\citep{royston1994}.
Interestingly, the figure implies three distinct growth regimes.
Those farms operating less than approximately 5.5 hectares appear to tend
to witness positive growth rates until reaching an equilibrium size of 5.5 
hectares.
Those farms operating between approximately 5.5 and 350 hectares appear
to tend to witness negative growth rates until arriving at that same 
equilibrium of 5.5 hectares.
Finally, those farms operating greater than approximately 350 hectares 
appear to tend to persistently grow.%
\footnote{While the estimates indeed imply continual growth among the 
latter regime, the number of observations in this regime is relatively few and 
more information may reveal a new equilibrium at the high end of the farm 
size spectrum.}
Most importantly, however, the figure conforms surprisingly well to the 
predictions of the model put forth by \citet{carter1993}, as the authors
contended that a dualistic agrarian structure is the likely product of the 
unfettered operation of the land market.

Table \ref{tests} also illustrates that titled area exerts influence on land
accumulation. 
Accordingly, Figure \ref{fig: experiment} depicts the relationship between 
the annual percentage growth rate in operational landholdings and initial land 
operated after a hypothetical land titling intervention.
The \emph{ex ante} curve is identical to that plotted in Figure 
\ref{fig: pgrowth} and is provided for purposes of comparison.
The \emph{ex post} curve was constructed analogously, but with one
simple change: the predicted values from the CR regression were calculated
after substituting the quantity of land owned for titled area.
This change reflects a situation in which a hypothetical property rights 
intervention provides formal title to all land owned by the surveyed
households.
Interestingly, the accumulation barriers remain identical to those of Figure
\ref{fig: pgrowth} (i.e.\ 5.5 and 350 hectares).
The hypothetical intervention, however, appears to have reduced the rate at
which the 5.5-350 ($>$350) hectare regime decumulates (accumulates) 
operational landholdings.
While a tendency toward dualism persists, property rights reform may 
thus possess the capacity to reduce the rate at which inequality 
manifests. 
Overall, then, the results associated with titled area conform reasonably well 
to the theoretical model put forth in \citet{carter1998b}.

%%%%%%%%%%%%%%%%%%%%%%%%%%%%%%
%%%%%%%%%%% Conclusions %%%%%%%%%%%%%
%%%%%%%%%%%%%%%%%%%%%%%%%%%%%%

\section{Conclusions}
\label{sec: conclusions}

To the extent that (1) inegalitarian distributions of agricultural landholdings in 
developing countries are a perceived source of economic inefficiency and (2) 
land rental and sales markets are the primary mechanism by which agricultural 
land is allocated and reallocated, an improved understanding of land 
accumulation dynamics is of considerable policy interest.
While theories of such accumulation dynamics are relatively numerous, 
empirical scrutiny of the associated hypotheses is relatively scarce.
With unique panel data from Paraguay, we thus employed a generalized 
method of moments (GMM) estimator for dynamic panel models in an 
effort to simultaneously test the leading theories.
As farm growth was found to vary systematically with select observable 
characteristics (i.e.\ land operated and titled area), we formally rejected
the theory of stochastic growth (i.e.\ Gibrat's Law.)
Moreover, the signs associated with the statistically significant effects 
empirically substantiated two of the hypotheses put forth: (1) that initial land 
operated is an important determinant of land accumulation 
\citep{carter1993} and (2) that titled area may exert a non-negligible 
influence on farm growth \citep{carter1998b}.

Interestingly, the estimates suggested that a dualistic agrarian structure is 
the likely product of the unfettered operation of land markets, though land 
titling interventions may possess the capacity to reduce the rate at which 
inequality manifests.
A thorough examination of the latter finding is beyond the scope of
the present analysis.
However, we speculate that, as tenure insecurity is often posited to relate
inversely with producer wealth and/or income, land titling interventions 
may disproportionately benefit small- and medium-sized agricultural 
producers.
An important caveat to this statement is that credit markets tend to 
persistently exclude asset-impoverished households regardless of the 
legal collateralizability of their land \citep{carter1988}, a consideration 
consistent with the finding that the smallest producers were little affected by 
our hypothetical property rights reform.
Thus, while land titling interventions may reduce the rate at which 
inequality manifests, it is possible that additional benefits could be realized if 
such interventions were accompanied by policies to improve the functioning
of credit markets.

Finally, although we consider our analysis to be an important advance in the 
understanding of land inequality in developing countries, it is important to 
recognize its limitations.
First, for reasons of data availability and institutional context, we focused
exclusively on the case of Paraguay. 
Second, even though the panel data utilized was unique in its length, the
number of producers in each cross section was relatively small.
Third, while great care was taken to illustrate that attrition bias is likely
an issue of negligible importance, we avoided undue complication of the 
estimation procedure and focused on surviving farms.
Future research may thus consider exploring similar research questions in 
other countries/settings, particularly with emphasis on wider panel data 
sets and adjustments for issues of attrition.
 
%%%%%%%%%%%%%%%%%%%%%%%%%%%%%%
%%%%%%%%%%%% References %%%%%%%%%%%%
%%%%%%%%%%%%%%%%%%%%%%%%%%%%%%

%Start fresh page
\newpage
\cleardoublepage
\singlespacing

%Declare the style of the bibliography
\bibliographystyle{au-cms}

%Specify the file to use
\bibliography{/Users/hendersonhl/Documents/References}
\newpage

%%%%%%%%%%%%%%%%%%%%%%%%%%%%%%
%%%%%%%%%%%% Appendix %%%%%%%%%%%%%
%%%%%%%%%%%%%%%%%%%%%%%%%%%%%%

\newpage
\doublespacing
\section*{Appendix}

%%%%% Distribution of farms by farm size   %%%%%%%

\footnotesize
\ctable[
cap = {Distribution of farms by farm size},
caption = {Distribution of farms by farm size},
captionskip = -2ex,
pos=htb,
label = {distf}
]{lrrrr}{
\tnote[a]{Data for the years 1956-1991 is from \citet{danielsen2009} and data 
for the year 2008 is from \citet{mag2012}.}
}{\hline \hline
& \multicolumn{1}{c}{1956} & 
 \multicolumn{1}{c}{1981} & 
\multicolumn{1}{c}{1991} & 
\multicolumn{1}{c}{2008} \\ \hline
 0-5 ha             & 45.9  & 36.0  & 40.0 &   40.5  \\[-1ex]
 5-10 ha               & 23.4  & 19.9  & 21.7 &   22.9 \\[-1ex]
 10-100 ha         & 27.4   & 40.0 & 34.3 &   30.2  \\[-1ex]
 100-500 ha         & 1.9    & 2.8    & 2.7 &      3.6  \\[-1ex]
 500-1,000 ha      &  0.4   & 0.4    & 0.5 &      0.9  \\[-1ex]
 1,000-10,000 ha  & 0.8   & 0.8    & 0.9  &     1.4  \\[-1ex]
 $>10,000$ ha    & 0.2    & 0.1    & 0.1  &    0.2   \\[-1ex]
 Total              & 100.0 & 100.0 & 100.0 & 100.0 \\[-1ex]
 Number of farms              & 149,614 & 248,930 & 307,221 & 289,649 \\ \hline
}
\normalsize

%%%%% Distribution of area by farm size   %%%%%%%

\footnotesize
\ctable[
cap = {Distribution of land by farm size},
caption = {Distribution of land by farm size},
captionskip = -2ex,
pos=htb,
label = {distl}
]{lrrrr}{
\tnote[a]{Data for the years 1956-1991 is from \citet{danielsen2009} and data 
for the year 2008 is from \citet{mag2012}.}
}{\hline \hline
& \multicolumn{1}{c}{1956} & 
 \multicolumn{1}{c}{1981} & 
\multicolumn{1}{c}{1991} & 
\multicolumn{1}{c}{2008} \\ \hline
 0-5 ha                & 1.0 &  0.7 & 1.0 & 0.8 \\[-1ex]
 5-10 ha              & 1.4 & 1.5  & 1.8 & 1.3 \\[-1ex]
 10-100 ha          &  5.0 & 9.5 & 9.1  & 5.7 \\[-1ex]
 100-500 ha        &  \rdelim\{{3}{8mm}[92.6]  & 6.3  & 6.8 & 7.4 \\[-1ex]
 500-1,000 ha      &   &  3.2  & 4.2   & 5.8 \\[-1ex]
 1,000-10,000 ha &  &  27.0 & 36.2 & 38.3 \\[-1ex]
 $>10,000$ ha    &    & 51.6 & 40.8  & 40.7 \\[-1ex]
 Total              & 100.0 & 100.0 & 100.0 & 100.0 \\[-1ex]
Land cultivated  & 16,816,618 & 21,940,531 & 23,817,737 & 31,086,894  \\ \hline
}
\normalsize
\newpage

%%%%%%%%%% Variable Definitions  %%%%%%%%%%
\footnotesize
\begin{longtable}{l  p{10.5cm}}
\caption[Variable definitions]{Variable definitions} \\[-2ex] 

%This is the header for the first page of the table...
\hline \hline \\[-5ex]
\multicolumn{1}{l}{Variable} &
\multicolumn{1}{c}{Definition}\\[0.0ex]   \hline
\endfirsthead

%This is the header for the remaining page(s) of the table...
\multicolumn{2}{c}{\normalsize{\tablename} \thetable{} -- Continued} \\
\hline \hline \\[-5ex]
\multicolumn{1}{l}{Variable} &
\multicolumn{1}{c}{Definition}\\[0.0ex]  
  \\[-4.5ex]
\endhead

%This is the footer for all pages except the last page of the table...
\multicolumn{2}{l}{{\textit{Continued on Next Page \ldots}}} \\
\endfoot

%This is the footer for the last page of the table...
  \\[-1.8ex]
\endlastfoot

% Content
Land Operated & Land owned plus land rented, sharecropped, or borrowed 
\emph{from} others less land rented, sharecropped, or borrowed \emph{to} 
others (hectares) \\

Titled Area & Quantity of land owned with legally registered, mortgageable 
property rights (hectares)\\

Labor  & Number of household members ages 15 to 64 \\

Dependents & Number of household members younger than 15 or older than 
64 years of age \\ 

Experience & Age of the household head less years of education of the 
household head less six years \\ 

Education & Years of education of the household head \\ \hline

\label{def}
\end{longtable}
\normalsize

%%%%%%%%% Descriptive Statistics   %%%%%%%%%

\footnotesize
\ctable[
cap = {Descriptive statistics},
caption = {Descriptive statistics},
captionskip = -2ex,
pos=htb,
label = {desc}
]{lrrrrrrrrrr}{
\tnote[a]{The variables experience and education have only 291 observations
for the year 1991.}
}{\hline \hline
& \multicolumn{2}{c}{1991} & \multicolumn{2}{c}{1994} & 
\multicolumn{2}{c}{1999} & \multicolumn{2}{c}{2002} & 
\multicolumn{2}{c}{2007} \\ \hline
Variable & \multicolumn{1}{c}{Mean} & \multicolumn{1}{c}{SD} &
\multicolumn{1}{c}{Mean} & \multicolumn{1}{c}{SD} &
\multicolumn{1}{c}{Mean} & \multicolumn{1}{c}{SD} &
\multicolumn{1}{c}{Mean} & \multicolumn{1}{c}{SD} &
\multicolumn{1}{c}{Mean} & \multicolumn{1}{c}{SD} \\ \hline
Land Operated & 32.16 & 72.30 & 38.29 & 86.87 & 34.82 & 83.33 & 38.24 & 
99.63 & 19.87 & 61.90\\[-1ex]
Titled Area       & 21.09 & 64.75 & 29.48 & 85.10 & 21.84 & 62.30 & 30.49 & 
91.93 & 11.05 & 42.71\\[-1ex]
Labor               &  2.93 &   1.55 &   2.61 &   1.35 &   2.88 &   1.73 &   3.22 &   
1.64 &  2.88 &   1.54\\[-1ex]
Dependents     &  2.85 &   2.08  &  2.45 &    2.01 &  2.40 &   1.94  &  2.35 &   
1.67 &  2.03 &   1.64\\[-1ex]
Experience      & 39.35 & 15.56 & 41.59 &  15.87 & 43.09 & 14.66 & 45.50 & 
15.08 & 42.88 & 15.77\\[-1ex]
Education        &  4.13 &   2.74 &   4.12 &   2.72  &  4.27 &   2.48 &   4.56 &   
2.65  &  4.73 &  2.86 \\[-1ex]
$N$       & \multicolumn{2}{c}{300} & \multicolumn{2}{c}{284} & 
\multicolumn{2}{c}{293} & \multicolumn{2}{c}{223} & 
\multicolumn{2}{c}{446} \\ \hline
}
\normalsize

%%%%%%%% Accumulation Graphic   %%%%%%%%

\begin{figure}[tbp]
\centering
\includegraphics[width=1\textwidth]{Land.pdf}
\caption{Mean farm size by cohort and year}
\label{fig: land}
\end{figure}

%%%%%Systems GMM Estimates %%%%%%%%%%

\footnotesize
\ctable[
	cap = {System GMM estimates},
	caption = {System GMM estimates},
	captionskip = -2ex,
	pos = htbp,
	label = sysgmm
]{lrrrrrr}{
	\tnote[a]{P-values $<$0.01, 0.05, and 0.10 correspond to $^{***}$, 
	$^{**}$, and $^{*}$, respectively.}
	\tnote[b]{The subscripts $1, 2, 3, 4$, and $5$ refer to titled area, labor, 
	dependents, experience, and education, respectively.}
	\tnote[c]{Standard errors are calculated using the 
	\citet{windmeijer2005} correction.}
}{ \hline \hline
& \multicolumn{2}{c}{FM} &\multicolumn{2}{c}{CM} 
&\multicolumn{2}{c}{CR} \\ \hline
Variable  & \multicolumn{1}{c}{Coeff.}  &  \multicolumn{1}{c}{SE} 
& \multicolumn{1}{c}{Coeff.} & \multicolumn{1}{c}{SE} 
& \multicolumn{1}{c}{Coeff.} & \multicolumn{1}{c}{SE}\\ \hline
$\ln y$ & -53.88* & 27.72 & -52.77* & 29.94 & -62.75** & 30.54\\[-1ex]
$\ln x_1$ & -2.09 & 19.56 & 8.92 & 17.89 & 16.98 & 20.02\\[-1ex]
$\ln x_2$ & 49.87 & 38.49 & 20.66 & 28.08 & 19.36 & 33.50\\[-1ex]
$\ln x_3$ & -1.20 & 31.38 & -13.61 & 36.41 & -11.28 & 40.23\\[-1ex]
$\ln x_4$ & 74.84 & 67.78 & -19.33 & 80.33 & 25.32 & 94.10\\[-1ex]
$\ln x_5$ & 76.21 & 46.92 & 22.81 & 54.62 & 41.03 & 58.32\\[-1ex]
$(\ln y)^2$ & 3.51** & 1.49 & 3.58 & 2.33 & 3.31* & 1.75\\[-1ex]
$\ln y \times \ln x_1$ & -5.97*** & 1.62 & -3.89** & 1.69 & -4.07** 
& 2.03\\[-1ex]
$\ln y \times \ln x_2$ & 0.54 & 3.42 & 0.95 & 3.89 & 1.03 & 3.31\\[-1ex]
$\ln y \times \ln x_3$ & 1.32 & 3.28 & 0.86 & 3.31 & 1.16 & 3.30\\[-1ex]
$\ln y \times \ln x_4$ & 8.17 & 5.73 & 7.59 & 5.87 & 9.68* & 5.81\\[-1ex]
$\ln y \times \ln x_5$ & 0.55 & 3.52 & -0.20 & 3.73 & 2.22 & 4.01\\[-1ex]
$(\ln x_1)^2$ & 5.67*** & 1.68 & 2.65 & 2.10 & 2.50 & 2.11\\[-1ex]
$\ln x_1 \times \ln x_2$ & 0.61 & 2.15 & -0.79 & 2.48 & -0.77 & 2.20\\[-1ex]
$\ln x_1 \times \ln x_3$ & 0.40 & 2.05 & -1.11 & 1.96 & -0.99 & 1.74\\[-1ex]
$\ln x_1 \times \ln x_4$ & 0.63 & 4.15 & 0.23 & 4.26 & -1.50 & 4.30\\[-1ex]
$\ln x_1 \times \ln x_5$ & -0.36 & 2.43 & -0.11 & 3.30 & -0.38 
& 3.68\\[-1ex]
$(\ln x_2)^2$ & 2.96 & 3.46 & 3.42 & 3.72 & 2.25 & 3.58\\[-1ex]
$\ln x_2 \times \ln x_3$ & -1.61 & 3.57 & -2.01 & 3.72 & -0.81 
& 3.74\\[-1ex]
$\ln x_2 \times \ln x_4$ & -11.87 & 9.02 & -4.03 & 6.18 & -3.21 
& 7.09\\[-1ex]
$\ln x_2 \times \ln x_5$ & -5.45 & 4.42 & -5.19 & 3.73 & -6.07 
& 4.75\\[-1ex]
$(\ln x_3)^2$ & 1.57 & 3.19 & 1.84 & 4.57 & 1.78 & 4.00\\[-1ex]
$\ln x_3 \times \ln x_4$ & 0.92 & 7.01 & 5.16 & 8.09 & 3.81 & 9.12\\[-1ex]
$\ln x_3 \times \ln x_5$ & -4.69 & 3.79 & -3.26 & 4.53 & -2.44 
& 4.62\\[-1ex]
$(\ln x_4)^2$ & -8.44 & 8.22 & 0.68 & 8.96 & -4.17 & 11.02\\[-1ex]
$\ln x_4 \times \ln x_5$ & -17.50 & 11.12 & -1.47 & 12.60 & -8.83 
& 13.63\\[-1ex]
$(\ln x_5)^2$ & -0.72 & 4.10 & -3.82 & 6.61 & -0.74 & 6.38\\[-1ex]
Time Dummies & \multicolumn{2}{c}{Yes} & \multicolumn{2}{c}{Yes}   
& \multicolumn{2}{c}{Yes}\\[-1ex]
Zero Value Dummies & \multicolumn{2}{c}{Yes} 
& \multicolumn{2}{c}{Yes}  & \multicolumn{2}{c}{Yes}\\[-1ex]
Instrument Count & \multicolumn{2}{c}{275} & \multicolumn{2}{c}{128}   
& \multicolumn{2}{c}{97}\\[-1ex]
$R^2$  & \multicolumn{2}{c}{0.10} & \multicolumn{2}{c}{0.10}  
&  \multicolumn{2}{c}{0.11} \\[-1ex]
$N$  & \multicolumn{2}{c}{820}   &  \multicolumn{2}{c}{820} 
&  \multicolumn{2}{c}{820} \\ \hline}
\normalsize
\newpage

%%%%%%%%%% Hypothesis Tests  %%%%%%%%%%%%

\footnotesize
\ctable[
	cap = {Hypothesis tests},
	caption = {Hypothesis tests},
	captionskip = -2ex,
	pos = htbp,
	label = tests
]{lrrrrrr}{
          \tnote[a]{P-values $<$0.01, 0.05, and 0.10 correspond to $^{***}$, 
          $^{**}$, and $^{*}$, respectively.}
	\tnote[b]{The subscripts $1, 2, 3, 4$, and $5$ refer to titled area, labor, 
	dependents, experience, and education, respectively.}
}{ \hline \hline
Hypothesis Test & \multicolumn{1}{c}{FM} &\multicolumn{1}{c}{CM}  
&\multicolumn{1}{c}{CR} \\ \hline
(1)~ Arellano-Bond                      &     1.10   &   0.93 &  1.05    \\[-1ex]
(2)~ Hansen                                 & 171.10   & 87.99 & 57.42 \\[-1ex]
(3)~ Difference-in-Hansen           &   19.07   & 35.12 & 25.82 \\[-1ex]
(4)~ $\partial/\partial \ln y$       &  -14.86***  & -10.16  & -10.74* \\[-1ex]
(5)~ $\partial/\partial \ln x_1$   &   14.83*** &   10.62** & 10.43**\\[-1ex]
(6)~ $\partial/\partial \ln x_2$   &     5.91* & 4.25 &  3.62 \\[-1ex]
(7)~ $\partial/\partial \ln x_3$    &    2.10  & 1.54 &   2.60   \\[-1ex]
(8)~ $\partial/\partial \ln x_4$    &    3.92  &  10.61 & 9.64 \\[-1ex]
(9)~ $\partial/\partial \ln x_5$    &  -0.58   & -3.38 &  3.66 \\ \hline}
\normalsize

%%%%%%%% Operated vs. Growth   %%%%%%%%

\begin{figure}[tbp]
\centering
\includegraphics[width=1\textwidth]{Predicted_Growth.pdf}
\caption{Predicted growth and land operated}
\label{fig: pgrowth}
\end{figure}

%%%%%%%%%% Experiment   %%%%%%%%%%

\begin{figure}[tbp]
\centering
\includegraphics[width=1\textwidth]{Predicted_Growth(Experiment).pdf}
\caption{Predicted growth pre- and post-reform}
\label{fig: experiment}
\end{figure}

\end{document}