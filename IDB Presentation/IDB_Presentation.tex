%%%%%%%%%%%%%%%%%%%%%%%%%%%%%%%%%%%%%
%%%%%%%%%%%%            PREAMBLE          %%%%%%%%%%%%
%%%%%%%%%%%%%%%%%%%%%%%%%%%%%%%%%%%%%

% Outline (25-30 slides):
% I. Introduction (2 slides) (Done)
% II. Theoretical background (6 slides) 
% III. Data (7-8 slides)
% IV. Methodology (7-8 slides)
% 	ADL model
% 	OLS, FD, and IV model
% 	Arellano-Bond estimator
% 	Blundell-Bond estimator
% 	Empirical model
% 	Hypothesis tests
% V. Results (3 slides) (Done)
% VI. Conclusions (1 slide) (Done)

\documentclass[xcolor=dvipsnames]{beamer} 
\usecolortheme[named=Black]{structure} 
\mode<presentation>{\usetheme{Classic}}
\usepackage[english]{babel}
\usepackage[latin1]{inputenc}
\usepackage{times}
\usepackage[T1]{fontenc}
\usepackage[authoryear]{natbib}
\usepackage{ctable}
\usepackage{setspace}
\usepackage{longtable}
\usepackage{mathrsfs}
\usepackage{amssymb}
\usepackage{mathtools}
\setbeamertemplate{caption}[numbered]
\setbeamerfont{caption}{size=\normalsize}
\setbeamercolor{caption name}{fg=black}
\setbeamercolor{item}{fg=black}
\setlength{\extrarowheight}{2.5pt}

%%%%%%%%%%%%%%%%%%%%%%%%%%%%%%%%%%%%
%%%%%%%%%%%%           TITLE PAGE         %%%%%%%%%%%
%%%%%%%%%%%%%%%%%%%%%%%%%%%%%%%%%%%%

\title[]{Land Accumulation Dynamics in Developing Country Agriculture}

\author[]{Heath Henderson, Leonardo Corral, Eric Simning \\
\emph{\footnotesize{
Inter-American Development Bank}} \\
~\\
Paul Winters \\
\emph{\footnotesize{American University}}}

\begin{document}
\def\newblock{\hskip .11em plus .33em minus .07em}

%%%%%%%%%%%%%%%%%%%%%%%%%%%%%%%%%%%%
%%%%%%%%%%%%%           CONTENT         %%%%%%%%%%%
%%%%%%%%%%%%%%%%%%%%%%%%%%%%%%%%%%%%

% Title slide
\begin{frame}
\titlepage
\end{frame}

% Introduction
\begin{frame}{Introduction}
\begin{itemize}
\item The distribution of landholdings in developing countries is a 
well-documented source of economic inefficiency.
\par\pause\noindent \item While documentation of the consequences of 
inequality is prevalent, the causes are less well understood.
\par\pause\noindent \item Theories have invoked stochastic growth 
processes, market imperfections, and/or institutional considerations, among 
others.
\par\pause\noindent \item With few exceptions such theories have not been 
subject to empirical scrutiny.
\par\pause\noindent \item Using unique panel data from Paraguay, we seek to 
test the leading theories of land accumulation dynamics.
\end{itemize}
\end{frame}

% Defining Land Accumulation
\begin{frame}{Defining Land Accumulation}
\begin{itemize}
\item Land accumulation is ``\textit{\textbf{the acquisition or gradual 
gathering of land use rights or land access for purposes of agricultural 
production}}.''
\par\pause\noindent \item This includes expansion of the farm unit via 
legal or extralegal means.
\par\pause\noindent \item As opposed to farm growth, land accumulation 
appears less ambiguous as farm growth can occur along multiple dimensions.
\par\pause\noindent \item The terms land accumulation and farm growth are, 
however, used synonymously throughout.
\end{itemize}
\end{frame}

% Outline
\begin{frame}{Outline}
\begin{enumerate}
\item Theoretical considerations
\item Paraguay background and data
\item Methodology
\item Results
\item Conclusions
\end{enumerate}
\end{frame}

% Theoretical Considerations
\begin{frame}{Theoretical Considerations}
\begin{enumerate}
\item Stochastic growth processes
\item Factor market imperfections
\item Institutional/legal considerations
\item Life cycle hypothesis
\item Managerial experience
\item Human capital
\end{enumerate}
\end{frame}

% Stochastic growth processes
\begin{frame}{Stochastic growth processes}
\begin{itemize}
\item The earliest and most influential of such theories is known as
``Gibrat's Law of Proportionate Effects'' (Gibrat, 1931).
\par\pause\noindent \item Let $x_t$ denote firm size at time $t$ and 
the random variable $\varepsilon_t$ denote the growth rate such that 
$x_t = x_{t-1}(1 + \varepsilon_t)$. 
\par\pause\noindent \item It follows that $\log(x_t) \approx 
\log(x_{0}) + \varepsilon_1 + \varepsilon_2 + \ldots + \varepsilon_t$. 
\par\pause\noindent \item Assuming $\varepsilon$ is i.i.d.\ with mean 
$\mu$ and variance $\sigma^2$, as $t \to \infty$ the distribution of 
$\log(x_t)$ is approximately normal.
\par\pause\noindent \item The central testable hypothesis is that firm 
growth is independent of initial firm traits, most notably firm size.
\end{itemize}
\end{frame}

% Factor Market Imperfections
\begin{frame}{Factor Market Imperfections}
\begin{itemize}
\item \citet{carter1993} focused on the role of labor and capital market 
imperfections in rural land markets.
\par\pause\noindent \item Given land endowments and market 
imperfections, agents choose their optimal time allocation and purchased 
inputs.
\par\pause\noindent \item The reservations price for $\varepsilon$ 
additional hectares of land is 
\begin{equation*}
\rho(T) = \sum_{t=1}^{\infty} \Delta_t(T)/[1 + \mu(T + \varepsilon)]^t
\end{equation*}
where $\Delta_t(T) = [\pi(T + \varepsilon) - \pi(T)]/\varepsilon $ and 
$\mu(T + \varepsilon)$ is the shadow price of capital.
\par\pause\noindent \item The central testable hypothesis is that farm 
size is a significant determinant of farm growth.
\end{itemize}
\end{frame}

% Institutional/Legal Considerations
\begin{frame}{Institutional/Legal Considerations}
\begin{itemize}
\item \citet{carter1998} incorporated tenure insecurity into a similar land 
market competitiveness model.
\par\pause\noindent \item Letting $0 < \phi < 1$ denote the probability 
that a household is dispossessed of its land, the reservation price is
\begin{equation*}
\rho(T) = \sum_{t=1}^{\infty} [(1 - \phi)^t \Delta_t(T, \phi)]/
[1 + \mu(T + \varepsilon, \phi)]^t
\end{equation*}
\par\pause\noindent \item Three channels: (1) uncertainty-based 
discounting; (2) factor allocations; and (3) credit supply effects.
\par\pause\noindent \item The central testable hypothesis is that tenure 
security is a significant determinant of land accumulation.
\end{itemize}
\end{frame}

% Life Cycle Hypothesis
\begin{frame}{Life Cycle Hypothesis}
\begin{itemize}
\item Chayanov (1966) suggested that land accumulation is intimately tied to 
the growth of the individual family. 
\par\pause\noindent \item Farm size adapts to the equilibrium income of 
the household, as determined by balancing the marginal utility of income 
($MU_I$) and marginal drudgery of labor ($MD_L$). 
\par\pause\noindent \item  $MU_I$ depends on family consumption 
demands and $MD_L$ depends on the size of the family work force.
\par\pause\noindent \item In traversing the family life cycle, these curves 
shift and influence farm size.
\par\pause\noindent \item The central testable hypothesis is that household 
labor and dependents are significant determinants of farm growth.
\end{itemize}
\end{frame}

% Managerial Experience
\begin{frame}{Managerial Experience}
\begin{itemize}
\item Jovanovic (1982) put forth a learning model where managerial 
experience is key to firm growth.
\par\pause\noindent \item Firms choose their output level $q_t$ to 
maximize expected profits $p_t q_t - c(q_t)x_t^*$ where $x_t^*$ is the 
expectation of $x_t$. 
\par\pause\noindent \item There is an equilibrium scale of production for 
mature firms as $x_t^*$ converges to a constant as firms age.
\par\pause\noindent \item While younger firms have more variable 
growth rates, Jovanovic demonstrated that they will grow faster.
\par\pause\noindent \item The central testable hypothesis is that there 
exists an inverse relationship between firm growth and firm age.
\end{itemize}
\end{frame}

% Human Capital
\begin{frame}{Human Capital}
\begin{itemize}
\item Rodgers (1994) emphasized the effects of human capital on land 
accumulation. 
\par\pause\noindent \item Let $x$ denote agriculture-specific human 
capital and $y$ denote general human capital.
\par\pause\noindent \item Given these endowments, agents choose to 
engage in agricultural production or off-farm employment.
\par\pause\noindent \item Off-farm income $w$ is an increasing 
function of $y$ and agricultural income $m$ is an increasing function 
of $x$.
\par\pause\noindent \item The central testable hypothesis is that human 
capital is a significant determinant of farm growth.
\end{itemize}
\end{frame}

% Paraguay Background
\begin{frame}{Paraguay Background}
\begin{itemize}
\item Paraguay, with a GDPPC of \$2,967 and a poverty rate of 38 percent, 
is among the poorest countries in Latin America.
\par\pause\noindent \item Agriculture accounts for 24 percent of GDP, 
27 percent of employment, and 87 percent of merchandise exports.
\par\pause\noindent \item With a Gini coefficient of 0.93, the distribution 
of land is one of the most inegalitarian in the world.
\par\pause\noindent \item Approximately 27 percent of producers do not 
have legal rights over the land they operate.
\end{itemize}
\end{frame}

% Paraguay Background
\begin{frame}{Paraguay Background}
\begin{itemize}
\item The concentration of landholdings is largely grounded in the aftermath 
of the War of the Triple Alliance (1864-1870).
\par\pause\noindent \item The Stroessner regime (1954-1989) enacted the 
Agrarian Statute of 1963 and embarked on a large-scale colonization program.
\par\pause\noindent \item Land issues became firmly established on the 
national political agenda after the fall of the Stroessner regime in 1989.
\par\pause\noindent \item The Constitution of 1992 and Agrarian Statute 
of 2002 left land redistribution to market forces and reflects the emergence 
of liberal democratic values.
\end{itemize}
\end{frame}

% Paraguay Background
\begin{frame}{Paraguay Background}
\footnotesize
\ctable[
cap = {Distribution of Farms by Farm Size},
caption = {Distribution of Farms by Farm Size},
captionskip = -1.5ex,
pos=htb,
label = {distfl}
]{lrrrr}{
\tnote[]{}
}{\hline \hline
& \multicolumn{1}{c}{1956} & 
 \multicolumn{1}{c}{1981} & 
\multicolumn{1}{c}{1991} & 
\multicolumn{1}{c}{2008} \\ \hline
0-5 ha             & 45.9  & 36.0  & 40.0 &   40.5  \\
5-10 ha               & 23.4  & 19.9  & 21.7 &   22.9 \\
10-100 ha         & 27.4   & 40.0 & 34.3 &   30.2  \\
100-500 ha         & 1.9    & 2.8    & 2.7 &      3.6  \\
500-1,000 ha      &  0.4   & 0.4    & 0.5 &      0.9  \\
1,000-10,000 ha  & 0.8   & 0.8    & 0.9  &     1.4  \\
$>10,000$ ha    & 0.2    & 0.1    & 0.1  &    0.2   \\
Total              & 100.0 & 100.0 & 100.0 & 100.0 \\
Number of farms  & 149,614 & 248,930 & 307,221 & 289,649 \\ \hline
}
\end{frame}

% LTC-CPES Data
\begin{frame}{LTC-CPES Data}
\begin{itemize}
\item In 1991, the Land Tenure Center (LTC) at the University of 
Wisconsin-Madison and the \emph{Centro Paraguayo de Estudios 
Sociol\'{o}gicos} (CPES) administered surveys to 300 rural Paraguayan 
households.
\par\pause\noindent  \item The households were selected in accordance 
with a multi-stage random sampling framework where households were 
stratified by region and farm size.
\par\pause\noindent \item The LTC-CPES survey is panel in nature 
and was further administered in the years 1994, 1999, 2002, and 2007. 
\par\pause\noindent \item The panel is unbalanced with 300, 284, 293, 
223, and 446 reliable observations across the five years, respectively. 
\end{itemize}
\end{frame}

% LTC-CPES Data
\begin{frame}{LTC-CPES Data}
\scriptsize
\ctable[
	cap = {Variable Definitions},
	caption = {Variable Definitions},
	captionskip = -1.5ex,
	pos = htbp,
	label = defs
]{ll}{
          \tnote[]{}
}{ \hline \hline
Variable & Definition \\ \hline
Land Operated ($y$) & Land owned plus land rented, sharecropped, or 
borrowed \\
& \emph{from} others less land rented, sharecropped, or borrowed 
\emph{to} \\
& others (ha) \\
Titled Area ($x_1$) & Quantity of land owned with legally registered, 
mortgageable \\
& property rights (ha)\\
Labor ($x_2$)  & Number of household members ages 15 to 64 \\
Dependents ($x_3$) & Number of household members younger than 15 or 
older \\
& than 64 years of age \\ 
Experience ($x_4$) & Age of the household head less years of education of 
the \\
& household head less six years \\ 
Education ($x_5$) & Years of education of the household head \\ \hline}
\end{frame}

% LTC-CPES Data
\begin{frame}{LTC-CPES Data}
\begin{figure}[htb]
    \centering
     \includegraphics[scale=0.60]{Land.pdf}
     \caption{Mean farm size by cohort and year}
\end{figure}
\end{frame}

% Methodology
\begin{frame}{Methodology}
\begin{itemize}
\item Consider the following autoregressive-distributed lag model: 
\begin{equation*}
y_{i,t} = \beta_1 y_{i,t-1} + \beta_2 x_{i,t} + \alpha_i +  u_{i,t}
\end{equation*}
for $i=1,2,\ldots, N$ and $t=2,3,\ldots,T$.
\par\pause\noindent \item The error term $u_{i,t}$ is assumed to be serially 
uncorrelated and independent across producers.
\par\pause\noindent \item OLS? The lagged dependent variable is correlated 
with $\alpha_i +  u_{i,t}$ so OLS is inconsistent.
\par\pause\noindent \item Within Groups? Let's see \ldots 
\end{itemize}
\end{frame}

% Methodology
\begin{frame}{Methodology}
\begin{itemize}
\item The transformed lagged dependent variable is
\begin{equation*}
y_{i,t-1} - \frac{1}{T-1}(y_{i,1} + \ldots + y_{i,t} + \ldots + y_{i,T-1})
\end{equation*}
\par\pause\noindent \item The transformed error term is
\begin{equation*}
u_{i,t} - \frac{1}{T-1}(u_{i,2} + \ldots + u_{i,t-1} + \ldots + u_{i,T})
\end{equation*}
\par\pause\noindent \item Due to the correlation between the two 
transformed terms, the Within Groups estimator is also inconsistent.
\par\pause\noindent \item What about first-differencing?
\end{itemize}
\end{frame}

% Methodology
\begin{frame}{Methodology}
\begin{itemize}
\item The first-difference model is as follows:
\begin{equation*}
\Delta y_{i,t} = \beta_1 \Delta y_{i,t-1} + \beta_2 \Delta x_{i,t} 
+ \Delta u_{i,t}
\end{equation*}
where $\Delta y_{i,t-1}= y_{i,t-1} - y_{i,t-2}$ and $\Delta u_{i,t} = 
u_{i,t} - u_{i,t-1}$ for $i=1,\ldots,N$ and $t=3,\ldots,T$.
\par\pause\noindent \item OLS estimation of the first-difference model 
is inconsistent.
\par\pause\noindent \item First-differencing does not introduce all 
realizations of the disturbances ($u_{i,2}, \ldots, u_{i,T}$) into the 
transformed error. 
\par\pause\noindent \item 2SLS? Yes, with $y_{i, t-2}$ as an instrument 
2SLS is consistent in large $N$, fixed $T$ panels 
(Anderson and Hsiao, 1982).
\end{itemize}
\end{frame}

% Methodology
\begin{frame}{Methodology}
\begin{itemize}
\item 2SLS is, however, not efficient as it generally does not utilize all 
available moment conditions.
\par\pause\noindent \item \citet{arellano1991} noted that for $T>3$ 
additional instruments are available (e.g.\ $y_{i, t-3}$ when $T=4$).
\par\pause\noindent \item The authors developed a GMM estimator in 
an effort to remedy the shortcomings of 2SLS.
\par\pause\noindent \item In the context of a simple autoregressive 
model (i.e.\ $\beta_2=0$), the instrument matrix can be written as follows:
\begin{equation*}
Z_i = \left[ 
\scriptsize
\begin{array}{ccccccc}
y_{i,1} & 0 & 0 & 0 & 0 & 0 & \cdots   \\\
0 & y_{i,1} & y_{i,2} & 0 & 0 & 0 & \cdots   \\\
0 & 0 & 0 & y_{i,1} & y_{i,2} & y_{i,3} & \cdots  \\\
\vdots & \vdots & \vdots & \vdots & \vdots & \vdots & \ddots
\end{array}
\right] 
\end{equation*}
\end{itemize}
\end{frame}

% Methodology
\begin{frame}{Methodology}
\begin{itemize}
\item $Z_i$ can also be ``collapsed'' as follows:
\begin{equation*}
\left[ 
\scriptsize
\begin{array}{ccccccc}
y_{i,1} & 0 & 0 & \cdots   \\\
y_{i,1} & y_{i,2} & 0 & \cdots   \\\
y_{i,1} & y_{i,2} & y_{i,3} & \cdots  \\\
\vdots & \vdots & \vdots & \ddots
\end{array}
\right]
\end{equation*}
\par\pause\noindent \item The above is readily generalized to the case 
of the autoregressive-distributed lag model.
\par\pause\noindent \item The efficient GMM estimator exploits the 
moment conditions $E[Z_i' \Delta u_i]=0$ for $i=1,2,\ldots,N$ to 
minimize
\begin{equation*}
J_N = \left(\frac{1}{N} \sum_{i=1}^N \Delta u_{i}^{\prime} Z_i \right) 
W_N \left(\frac{1}{N} \sum _{i=1}^N Z_i' \Delta u_i\right)
\end{equation*}
\end{itemize}
\end{frame}

% Methodology
\begin{frame}{Methodology}
\begin{itemize}
\item \citet{blundell1998} noted that when the series in question has near 
unit root properties, the utilized instruments are likely to be weak.
\par\pause\noindent \item Past changes may be more predictive of current 
levels than past levels are of current changes.
\par\pause\noindent \item The authors suggested utilizing the additional 
$T-2$ moment conditions $E[\Delta y_{i,t-1} (\alpha_i + u_{i,t})]=0$ 
for $i=1,2,\ldots,N$ and $t=3,4,\ldots,T$.
\par\pause\noindent \item The ``system'' GMM estimator stacks the 
$T-2$ equations in first-differences and the $T-2$ equations in levels.
\end{itemize}
\end{frame}

% Methodology
\begin{frame}{Methodology}
\begin{itemize}
\item For the simple autoregressive model, the augmented instrument matrix is 
as follows:
\begin{equation*}
Z_{i}^{+} = \left[ 
\scriptsize
\begin{array}{ccccccc}
Z_i  & 0 & 0 & 0 & \ldots \\
0 & \Delta y_{i,2} & 0 & 0 & \ldots \\
0 & 0 & \Delta y_{i,3} & 0 & \ldots  \\
0 & 0 & 0 & \Delta y_{i,4} & \ldots  \\
\vdots & \vdots & \vdots & \vdots & \ddots
\end{array}
\right] \text{or}
\left[ 
\scriptsize
\begin{array}{ccccccc}
Z_i  & 0  \\
0 & \Delta y_{i,2}  \\
0 & \Delta y_{i,3}  \\
0 & \Delta y_{i,4}  \\
\vdots & \vdots 
\end{array}
\right]
\end{equation*}
\par\pause\noindent \item The above is also readily generalized to the 
case of the autoregressive-distributed lag model.
\par\pause\noindent \item Estimation then consists of minimizing 
$J_N$ after properly introducing the additional moment conditions. 
\end{itemize}
\end{frame}

% Methodology
\begin{frame}{Methodology}
\begin{itemize}
\item The following general specification is put forth here:
\begin{equation*}
\frac{\ln(y_{i,t}) - \ln(y_{i,t-1})}{\delta_t} \times 100 =  
\ln [g(y_{i,t-1}, x_{i,t-1})]+ \alpha_i  + u_{i,t}
\end{equation*}
\par\pause\noindent \item Approximating the growth function $g(\cdot)$ 
by a second-order expansion in the logs, the above can be estimated with the 
``system'' GMM estimator.
\par\pause\noindent \item Four assumptions: (1) serial correlation; (2) 
instrumental exogeneity; (3) instrument count; and (4) attrition bias. 
\end{itemize}
\end{frame}

% Results
\begin{frame}
\footnotesize
\ctable[
	cap = {Hypothesis Tests},
	caption = {Hypothesis Tests},
	captionskip = -1.5ex,
	pos = htbp,
	label = tests
]{lrrrrrr}{
          \tnote[]{Note: P-values $<$0.01, 0.05, and 0.10 correspond to $^{***}$, 
          $^{**}$, and $^{*}$, respectively. The subscripts $1, 2, 3, 4$, and $5$ 
          refer to titled area, labor, dependents, experience, and education, 
          respectively. Marginal effects are calculated at sample means.}
}{ \hline \hline
Hypothesis Test & \multicolumn{1}{c}{FM} &\multicolumn{1}{c}{CM}  
&\multicolumn{1}{c}{CR} \\ \hline
(1)~ Arellano-Bond                      &     1.10   &   0.93 &  1.05    \\
(2)~ Hansen                                 & 171.10   & 87.99 & 57.42 \\
(3)~ Difference-in-Hansen           &   19.07   & 35.12 & 25.82 \\
(4)~ $\partial/\partial \ln y$       &  -14.86***  & -10.16  & -10.74* \\
(5)~ $\partial/\partial \ln x_1$   &   14.83*** &   10.62** & 10.43**\\
(6)~ $\partial/\partial \ln x_2$   &     5.91* & 4.25 &  3.62 \\
(7)~ $\partial/\partial \ln x_3$    &    2.10  & 1.54 &   2.60   \\
(8)~ $\partial/\partial \ln x_4$    &    3.92  &  10.61 & 9.64 \\
(9)~ $\partial/\partial \ln x_5$    &  -0.58   & -3.38 &  3.66 \\ \hline}
\normalsize
\clearpage
\end{frame}

% Results
\begin{frame}{Results}
\begin{figure}[htb]
    \centering
     \includegraphics[scale=0.60]{Predicted_Growth.pdf}
     \caption{Predicted growth and land operated}
\end{figure}
\end{frame}

% Results
\begin{frame}{Results}
\begin{figure}[htb]
    \centering
     \includegraphics[scale=0.60]{Predicted_Growth(Experiment).pdf}
     \caption{Predicted growth pre- and post-reform}
\end{figure}
\end{frame}

% Conclusions
\begin{frame}{Conclusions}
\begin{itemize}
\item The results suggested that farm growth varies systematically with select 
observable characteristics, implying a rejection of Gibrat's Law.
\par\pause\noindent \item The signs associated with the statistically 
significant effects empirically substantiated two of the hypotheses put forth.
\par\pause\noindent \item The estimates suggested that a dualistic 
agrarian structure is the likely product of the unfettered operation of land 
markets.
\par\pause\noindent \item Land titling interventions may possess the 
capacity to reduce the rate at which inequality manifests.
\end{itemize}
\end{frame}

% References
\begin{frame}{Select References}
\bibliographystyle{au-cms}
\bibliography{/Users/hendersonhl/Documents/References}
\end{frame}

\end{document}